\documentclass[]{article}
\usepackage{threeparttable}
\usepackage{booktabs}
\usepackage{dcolumn}
\newcolumntype{L}[1]{>{\raggedright\let\newline\\\arraybackslash\hspace{0pt}}m{#1}}
\newcolumntype{C}[1]{>{\centering\let\newline\\\arraybackslash\hspace{0pt}}m{#1}}
\newcolumntype{R}[1]{>{\raggedleft\let\newline\\\arraybackslash\hspace{0pt}}m{#1}}
\newcolumntype{.}{D{.}{.}{-1}}
\usepackage{amsmath}
\usepackage{pdflscape}
\usepackage{array}
\usepackage{fancyhdr}
\usepackage{lastpage}
\usepackage{textcomp}
\usepackage[T1]{fontenc}
\usepackage{caption}
\usepackage{subcaption}
\usepackage{graphicx}
\usepackage[margin=0.8in, bottom=1.2in]{geometry}
\usepackage[page]{appendix}
\pagestyle{fancy}
\renewcommand{\headrulewidth}{0pt}
\fancyhead{}
\rfoot{Page \thepage\  of \pageref{LastPage}}
\cfoot{}
\begin{document}
\begin{table}
\centering
\begin{threeparttable}
\caption{My Table Title}
\begin{tabular}{lccc}
\toprule
  & a & b & c\\
\midrule
\multicolumn{4}{l}{Panel A: One}\\
0 &  1 &  2 &  3 \\
1 &  4 &  5 &  6 \\
2 &  7 &  8 &  9 \\
  &   &   &  \\
\multicolumn{4}{l}{Panel B: Two}\\
0 &  11 &  12 &  13 \\
1 &  14 &  15 &  16 \\
2 &  17 &  18 &  19 \\
\bottomrule

\end{tabular}
\begin{tablenotes}[para, flushleft]
My below text
\end{tablenotes}
\end{threeparttable}
\end{table}
\end{document}