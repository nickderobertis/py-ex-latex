\documentclass[]{article}
\usepackage{amsmath}
\usepackage{pdflscape}
\usepackage{booktabs}
\usepackage{array}
\usepackage{threeparttable}
\usepackage{fancyhdr}
\usepackage{lastpage}
\usepackage{textcomp}
\usepackage{dcolumn}
\newcolumntype{L}[1]{>{\raggedright\let\newline\\\arraybackslash\hspace{0pt}}m{#1}}
\newcolumntype{C}[1]{>{\centering\let\newline\\\arraybackslash\hspace{0pt}}m{#1}}
\newcolumntype{R}[1]{>{\raggedleft\let\newline\\\arraybackslash\hspace{0pt}}m{#1}}
\newcolumntype{.}{D{.}{.}{-1}}
\newcolumntype{d}[1]{D{.}{.}{#1}}
\usepackage[T1]{fontenc}
\usepackage{caption}
\usepackage{subcaption}
\usepackage{graphicx}
\usepackage[margin=0.8in, bottom=1.2in]{geometry}
\usepackage[page]{appendix}
\usepackage{siunitx}
\pagestyle{fancy}
\renewcommand{\headrulewidth}{0pt}
\fancyhead{}
\rfoot{Page \thepage\  of \pageref{LastPage}}
\cfoot{}
\begin{document}
\begin{table}
\centering
\begin{threeparttable}
\caption{My Table Title}
\begin{tabular}{lS[table-column-width=4cm]s}
\toprule
a & b & c\\
1 &  2.540 &   3.100 \\
4 & 50.000 & 654.000 \\
7 &  8.114 &   9.541 \\
\bottomrule

\end{tabular}
\begin{tablenotes}[para, flushleft]
My below text
\end{tablenotes}
\end{threeparttable}
\end{table}
\end{document}